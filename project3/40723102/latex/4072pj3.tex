\documentclass[14pt,a4paper]{report}  %紙張設定
\usepackage{xeCJK}%中文字體模組
\setCJKmainfont{標楷體} %中文字體
\usepackage{amsmath,amssymb}%數學公式、符號
\usepackage{graphicx, subfig}%圖形
\usepackage{graphicx, subfig}%圖片模組
\usepackage{type1cm} %調整字體絕對大小
\usepackage{textpos} %設定文字絕對位置
\usepackage[top=2.5truecm,bottom=2.5truecm,
left=3truecm,right=2.5truecm]{geometry}
\usepackage{titlesec} %目錄標題設定模組
\usepackage{titletoc} %目錄內容設定模組
\usepackage{caption} %圖片標題設定
\usepackage{CJK} %中文模組
\usepackage{CJKnumb} %中文數字模組
%\usepackage{wallpaper} %浮水印
\usepackage{float}
\usepackage{enumerate}
\usepackage{listings} %引用程式碼
\usepackage{indentfirst}

\lstset{language=Python, %設定語言
		basicstyle=\footnotesize, 
		frame=lines,
		caption={Python code}, %設定capion
		label={lst:example} %設定label
		}
\renewcommand{\baselinestretch}{1.0} %設定行距1.0

\newcommand{\thirty}{\fontsize{30pt}\baselineskip\selectfont}%字體大小30pt
\newcommand{\twentyfour}{\fontsize{24pt}{\baselineskip}\selectfont}%字體大小24pt
\newcommand{\twenty}{\fontsize{20pt}{\baselineskip}\selectfont}%字體大小20pt
\newcommand{\eigh}{\fontsize{18pt}{\baselineskip}\selectfont}%字體大小18pt
\newcommand{\sixteen}{\fontsize{16pt}{\baselineskip}\selectfont}%字體大小16pt
\newcommand{\fourteen}{\fontsize{14pt}{\baselineskip}\selectfont}%字體大小14pt
\newcommand{\twelve}{\fontsize{12pt}{\baselineskip}\selectfont}%字體大小12pt

%設定章節標題內容
\titleformat{\chapter}[hang]{\center \twenty \bf}{\twenty 第\CJKnumber{\thechapter}章}{1em}{}[]
\titleformat{\section}[hang]{\raggedright \eigh \bf}{\eigh 第{\thesection}節}{1em}{}[]
\titleformat{\subsection}[hang]{\raggedright \sixteen \bf}{\sixteen {\thesubsection}}{1em}{}[]
%章節間距
\titlespacing*{\chapter} {0pt}{0pt}{18pt}
\titlespacing*{\section} {0pt}{12pt}{6pt}
\titlespacing*{\subsection} {0pt}{6pt}{6pt}
%設定目錄內容
\titlecontents{chapter}[11mm]{}{\normalfont\fontsize{14pt}{2.5pt}\bfseries\makebox[3.5em][l]
{第\CJKnumber{\thecontentslabel}章}}{}{\titlerule*[0.7pc]{.}\contentspage}
\titlecontents{section}[18mm]{}{\fourteen \makebox[2em][l]
{\fourteen \thecontentslabel}}{}{\titlerule*[0.7pc]{.}\contentspage}
\titlecontents{subsection}[25mm]{}{\fourteen \makebox[2.5em][l]{{\fourteen \thecontentslabel}}}{}{\titlerule*[0.7pc]{.}\contentspage}
%設定圖表目錄內容
\newcommand{\loflabel}{圖}
\newcommand{\lotlabel}{表}
\captionsetup{font={Large}} %設定標題大小
\captionsetup[figure]{name=圖} %固定標題名稱

\begin{document}
    {\renewcommand\baselinestretch{1.4}\selectfont %設定以下行距
    {\begin{center} %以下文字置中
        \twentyfour{國立虎尾科技大學}\\{機械設計工程系}\\{專題製作報告}\\
        \hspace*{\fill} \\
        \bf \thirty {Pyslvs-UI 平面多連桿機構套件之合成與應用}\\
        {\renewcommand\baselinestretch{1.3}\selectfont %設定以下行距
        \bf \thirty {Synthesis and Application of Pyslvs-UI Planar Multi-link Mechanism Package}\\}
    \end{center}}
    \par} %結束指定行距

     {\renewcommand\baselinestretch{1.2}\selectfont %設定以下行距
    {\begin{textblock}{30}(1.45,1.1) %{寬度}(以左上角為原點之右移量,下移量)
    \noindent \eigh \makebox[6em][s]{指導教授}\enspace:\qquad
    \eigh \makebox[8em][s]{李武鉦}\\ %\noindent指定首行不進行縮排
    \eigh \makebox[6em][s]{班級}\enspace:\qquad
    \eigh \makebox[8em][s]{四設計三甲} \\ %\makebox為文本盒子
    \eigh \makebox[6em][s]{學生}\enspace:\qquad
    {\begin{minipage}[t][9.8em][s]{12em}
    \eigh \setlength{\baselineskip}{10pt plus 2pt}
    林昱秀\qquad40723102\vfill 林晏瑩\qquad40723103\vfill 劉光智\qquad40723145\vfill 吳佳穎\qquad40723153\vfill 蔡育灃\qquad40723245
    \end{minipage}}
    \end{textblock}} %指定以下文字整體偏移至頁面的絕對位置
    \par} %結束指定行距

    {\begin{textblock}{12}(0,6.8)
    {\begin{center}
    \noindent \eigh \makebox[12em][s]{中華民國一一零年四月}
    \end{center}}
    \end{textblock}}
    \newpage
%專題製作合可證明
 {\renewcommand\baselinestretch{1.4}\selectfont %設定以下行距
 {\begin{center}
    {\eigh {國立虎尾科技大學 \qquad 機械設計工程系}\\{學生專題製作合格認可證明}\\
    \hspace*{\fill} \\ %似enter鍵換行
    \par}
     \end{center}}
    {\begin{textblock}{60}(1.85,0.8)
    \noindent \fourteen 專題製作修習學生\enspace:\quad
    {\begin{minipage}[t]{10em}\underline{四設三甲\enspace 40723102\enspace 林昱秀}\\ \underline{四設三甲\enspace 40723103\enspace 林晏瑩}\\ \underline{四設三甲\enspace 40723145\enspace 劉光智}\\ \underline{四設三甲\enspace 40723153\enspace 吳佳穎}\\ \underline{四設三乙\enspace 40723245\enspace 蔡育灃}\\ %下劃線符號指令
    \end{minipage}}
         \par} %結束指定行距
    {\renewcommand\baselinestretch{1.2}\selectfont %設定以下行距
    {\begin{textblock}{30}(1.8,4)
    \noindent \fourteen 專題製作題目\enspace :\quad Pyslvs-UI 平面多連桿機構套件之合成與應用
    \hspace*{\fill} \\
    \hspace*{\fill} \\
    \noindent \fourteen 經評量合格,特此證明
    \hspace*{\fill} \\
    \hspace*{\fill} \\
    \noindent \fourteen \makebox[6em][s]{評審委員}\enspace:\quad
    {\begin{minipage}[t]{6em} \underline{            }\\ \underline{            }\\ \underline{            }\\
    \end{minipage}}
    \end{textblock}}
    {\begin{textblock}{10}(1.8,9)
    {\begin{flushleft}
    \fourteen \makebox[6em][s]{指導老師}\enspace:\quad \underline{            }\\
    \fourteen \makebox[6em][s]{系主任}\enspace:\quad \underline{            }\\
    \hspace*{\fill} \\
    \fourteen \makebox[12em][s]{中華民國一一零年}
    \fourteen \makebox[9em][s]{四月二十九日}
    \end{flushleft}}
    \end{textblock}}
    \end{textblock}}
     \par} %結束指定行距
     \newpage
%摘要
	\pagenumbering{roman} %設定以下頁碼為羅馬數字
    \addcontentsline{toc}{chapter}{\fourteen{摘要}} %將摘要加入目錄
    \begin{center}
    \twenty \bf{摘要}\\
    \end{center}
	\hspace*{\fill} \\

	\quad \fourteen {本專題利用一開源套件Pyslvs-UI進行機構尺寸合成,並以登山車避震機構作為範例,針對影響登山車騎乘舒適度及整體平衡進行防後蹲(Anti-squat)及防後升(Anti-rise)做為設計需求建立適應函數再將合成所得結果進行量化評比並探討不同避震機構類型的差異,後續將合成後的尺寸作為參數套入自行編寫的程式透過Inventor自動生成3D零件,接著將上述所建立的零件利用xml格式設定零件組合位置、零件從屬關係和動態系統的碰撞設定後匯入CoppeliaSim,最後以CoppeliaSim API控制馬達轉軸讓機構進行運動模擬。}
    
	\begin{flushleft}
	\fontsize{14pt}{2.5pt}{關鍵字:機構合成、機構分析、登山車避震機構}
	\end{flushleft}
%目錄
\addcontentsline{toc}{chapter}{\fourteen{目錄}} %將目錄加入目錄
{\begin{center}
\renewcommand{\contentsname}{\centerline{\fontsize{20pt}{\baselineskip}\selectfont\textbf{目\quad 錄}}}
{\fourteen \tableofcontents}
%圖表目錄
\addcontentsline{toc}{chapter}{\fourteen{圖表目錄}} %將圖表目錄加入目錄
\renewcommand{\listfigurename}{\centerline{\fontsize{20pt}{\baselineskip}\selectfont\textbf{圖\quad 表\quad 目\quad 錄 }}}
\renewcommand{\numberline}[1]{\loflabel~#1\hspace*{1em}}
{\fourteen \listoffigures}
\end{center}}
         
	{\begin{center}
     \chapter{簡介}
     \end{center}}
      \section{研究背景與動機}
      \quad \fourteen {隨著科技越來越發達網路上有許多機構分析的套件,但是鮮少有機構合成功能的套件,且大多無公開原始碼。而本專題所使用的套件-Pyslvs-UI為本研究室所開發的套件且有公開原始碼,因此本專題利用該套件驗證合成及模擬功能並以合成一登山車後避震機構為例。}
	  \\
      \section{研究目的與方法}
      \quad \fourteen {首先建立一現有登山車避震機構,再針對不影響現有自行車車架的尺寸但對於避震能力有較大影響的接頭利用遺傳演算法求解最佳位置以得到最佳連桿尺寸,接著將上述所得到的接頭位置和連桿長度利用Inventor自動產生對應連桿尺寸的零件,後續利用xml格式匯入CoppeliaSim並利用CoppeliaSim API設定馬達讓該機構進行運動模擬。上述流程如下圖所示。}
     

    {\begin{center}
        \chapter{文獻探討}
        \end{center}}
      \section{平面機構分析套件}
       	\hspace*{\fill} \\
       	\subsection{Linkage}
      \quad \fourteen {Linkage為電腦輔助設計的應用程式,用於機械鏈接的原型製作。基本功能為設計帶有樞軸連接器或滑動連接器之機構、驅動機構的輸入可以是旋轉或線性的、可在運動的連桿上安裝齒輪和鏈條機構、將繪圖功能分配給連接點顯示其運動之軌跡。}
 		\begin{figure}[H]
        \centering
        \includegraphics[scale=1]{Linkage.png} 
        \caption{Linkage工作介面} 
        \label{fig:scale}
    	\end{figure}
	  	\hspace*{\fill} \\
\newpage
	  	\subsection{M.Sketch}
       	\hspace*{\fill} \\        
       \fourteen {M.Sketch,一種原型設計工具,設計目的在為非專業人士提供簡單直觀的界面,輕鬆學習與使用幾何圖形的模擬,如圖2.2。其中有五個主要功能:}
       \begin{itemize} 
       \item 繪製機構草圖:軟體介面中,藉由單擊與拖動滑鼠來簡單的繪製連桿接點,也可將接點轉換為滑塊,以此快速設計連桿機構。
       \item 運動軌跡可視化:根據馬達的旋轉值,可看到動畫上每個鏈接的坐標變化、設定欲關注之端點並給予標記,則可看到該端點預期的運動路徑。
       \item 逐步設計:首先根據繪製所需之運動機構後,使其生成機構路徑,再優化當前機構所適合的路徑。
       \item 加工支援:可導出PDF文件中每個鏈接的組合圖、示意圖和零件圖,亦可導出鏈接和關聯零件的3D模型,以匯入3D列印機等。
       \item 高可及性:M.Sketch是作為Web的應用程序開發的,用戶可以在Web瀏覽器(例如GoogleChrome)上申請帳號後,即可直接使用,無需在電腦上進行任何安裝。
\end{itemize}  
       
 		\begin{figure}[H]
        \centering
        \includegraphics[scale=0.25]{M.Sketch.jpg} 
        \caption{M.Sketch工作介面} 
        \label{fig:scale}
    	\end{figure}
\hspace*{\fill} \\
\newpage
      \section{平面機構合成套件}
      	\subsection{MeKin2D}
       		\hspace*{\fill} \\
       	\fourteen {MeKin2D是用Free Pascal編寫的子程序包,主要使用模組化方法對平面連桿機構進行運動學模擬,用於凸輪機構的合成和分析,以及漸開線齒輪的產生。MeKin2D的獨特功能包括:}
       \begin{itemize}
       \item 具有生成重複機構的能力。
       \item 將速度和加速度(法向和切向)顯示為向量。
       \item 將多邊形形狀附加到運動的接點。
       \item 顯示動態時線性和角度的大小。
       \item 顯示螺桿實際運作時的運動軌跡。
       \item 可以合成與分析凸輪機構。
\end{itemize}        
       	\begin{figure}[H]
        	\centering
        	\includegraphics[scale=0.4]{平面連桿之運動學模擬.png} 
        	\caption{平面連桿之運動學模擬} 
        	\label{fig:scale}
        	\includegraphics[scale=0.8]{盤式凸輪機構之合成與分析.png} 
        	\caption{盤式凸輪機構之合成與分析} 
        	\label{fig:scale}   
    	\end{figure}
		 	\hspace*{\fill} \\
       \subsection{WinMecC}
       		\hspace*{\fill} \\        
        \fourteen {WinMecC為一種電腦軟件,主要用於一自由度平面機構的分析與合成。基本功能為將機構的運動學和動力學分析之結果數值化或圖視化、可以優化分析創建的機構,依據特定點所期望遵循之路徑進行機構合成,如圖2.4所示。}
        \begin{figure}[H]
        \centering
        \includegraphics[scale=0.6]{WinMecC.png} 
        \caption{WinMecC} 
        \label{fig:scale}
    	\end{figure}
    	 \hspace*{\fill} \\
       \subsection{GIM}
       	\hspace*{\fill} \\        
        \fourteen {GIM是COMPMECH Research Group創建的註冊免費軟件,用於平面機構的運動模擬合成以及機械結構的靜態分析。GIM軟件具有三個主要模組:幾何、運動、合成。}
        \begin{itemize} 
       \item 幾何模組:設計要分析的機構,建立其節點與元件,定義連接點的類型(旋轉
接頭或滑塊,並設定固定點。
       \item 運動模組:透過運動學分析與模擬,在運動點繪製路徑以及路徑的曲率中心,可在畫面上看到運動極點的位置、速度和加速度向量,並繪製與分析數據相關的圖形和表格。
       \item 合成模組:依據特定點所期望之運動路徑,進行機構軌跡合成。如圖2.5所示。
       \end{itemize}
       
        \begin{figure}[H]
        \centering
        \includegraphics[scale=0.8]{GIM.png} 
        \caption{GIM合成介面} 
        \label{fig:scale}
    	\end{figure}
      \newpage
      \section{自行車避震機構研究}
       \subsection{Single-pivot suspension(單軸懸架)}
       \hspace*{\fill} \\
       \fourteen {最簡單的懸架設計為單樞軸,使用搖臂連接後軸、主樞軸、避震器,特點在於後軸直接連接至主樞軸,避震器連接到搖臂上,搖桿環繞在主樞軸中心旋轉,槓桿比由避震器的位置決定,固定中心可在行駛過程預測懸架特性,但也意味著不同階段修改懸架特性能力有限,示意圖產品型號為 Orange Stage的 STAGE 6 FACTORY,如圖(2.1)。}
     \begin{figure}[H]
        \centering
        \includegraphics[scale=0.8]{單軸懸架.jpg} 
        %照片須放在和文檔相同之資料夾內
        \caption{單軸懸架} %有caption的圖才會編入目錄中
        \label{fig:scale} %此處的label相當於一個圖片的專屬標誌,目的是方便上下文的引用
    \end{figure}
       \hspace*{\fill} \\
       \subsection{Linkage-driven Single-pivot suspension(連桿驅動之單樞軸懸架)}
       \hspace*{\fill} \\        
        \fourteen {如同單樞軸設計藉由後軸通過搖臂連接到主樞軸,增加聯動裝置在避震與搖臂間允許調整槓桿比率曲線,並在搖臂與避震器間建立連桿以改變槓桿比,搖臂圍繞在主軸旋轉且將中心固定於整個行程中,示意圖產品車架為Orange Stage的Orange 2020 Alpine 6 Frame 27.5,如圖(2.2)。}
     \begin{figure}[H]
        \centering
        \includegraphics[scale=0.65]{連桿驅動之單樞軸懸架.jpg} 
        %照片須放在和文檔相同之資料夾內
        \caption{連桿驅動之單樞軸懸架} %有caption的圖才會編入目錄中
        \label{fig:scale} %此處的label相當於一個圖片的專屬標誌,目的是方便上下文的引用
    \end{figure}
        \newpage
        \subsection{High-Pivot Idler suspension(高樞轉惰輪懸架)}
        \hspace*{\fill} \\
        \fourteen{通常建構在單樞軸或鏈接驅動的單樞軸設計上,其樞軸位置位於鏈條上方,為使控制鏈條增長,鏈輪惰輪將鏈條從鏈輪佈置到主樞軸的頂部或非常靠近主樞軸的位置,主樞軸定位在鏈輪上方更高的位置,使用單樞軸同時中心在整個行程中是固定的。
         \hspace*{\fill} \\  
但是,在撞擊過程中,後搖臂在主樞軸下方以曲線向上和向後旋轉,而不是在低樞軸位置向上和向後旋轉,且後軸路徑可以幫助懸架在方形撞擊中保持平穩。
		 \hspace*{\fill} \\  
鏈條繞到惰輪上,使鏈條與樞軸和後軸成一直線,大大減少了鏈增長的影響。惰輪的位置可以由設計人員用來增加或減少防下蹲的水平,但不會改變通常較高的防起落高度,如圖(2.3),產品來源為Forbidden Druid的Druid Frame – 2021。}
     \begin{figure}[hbt!]
        \centering
        \includegraphics[scale=0.6]{高樞轉惰輪懸架.jpg} 
        %照片須放在和文檔相同之資料夾內
        \caption{高樞轉惰輪懸架} %有caption的圖才會編入目錄中
        \label{fig:scale} %此處的label相當於一個圖片的專屬標誌,目的是方便上下文的引用
    \end{figure}  
       \hspace*{\fill} \\
        \subsection{Twin-link suspension(雙連桿懸掛)}
        \hspace*{\fill} \\
         \fourteen {搖臂通過兩個搖桿連桿安裝在框架上,從而將搖臂連接到主機樞軸,通過擺臂或搖臂連桿之一來驅動衝擊。
         \hspace*{\fill} \\  
         雙連桿懸架將樞軸的數量從一增加到四個,可修改中心位置,進而改變行程不同點的懸架特性,參考產品為Ibis’ DW-link Suspension的Mojo HD5 Frame – 2020,如圖(2.4)。}
     \begin{figure}[hbt!]
        \centering
        \includegraphics[scale=0.7]{雙連桿懸掛.jpg} 
        %照片須放在和文檔相同之資料夾內
        \caption{雙連桿懸掛} %有caption的圖才會編入目錄中
        \label{fig:scale} %此處的label相當於一個圖片的專屬標誌,目的是方便上下文的引用
    \end{figure}
    \newpage  
    	\hspace*{\fill} \\
        \subsection{Horst-link suspension(霍斯特鏈懸掛)}
        \hspace*{\fill} \\
         \fourteen {特性在於樞軸底端的鏈條,後軸則安裝在腳撐上增加了樞軸,其修改後的軸距繞即時中心旋轉,改變行進路線的位置,可在行進的不同階段優化防下蹲與上抬。
         \hspace*{\fill} \\  
雙連桿懸掛相比,霍斯特連桿系統的較長連桿通常會提供更平滑的曲線,產品支架為RAAW Madonna的MADONNA V1 - FAST AND PREDICTABLE,如圖(2.5),產品為 Canyon Spectral的Izzo Pro Race 29,如圖(2.6)。}
     
     \begin{figure}[hbt!]
        \centering
        \includegraphics[scale=0.2]{霍斯特鏈懸掛2.jpg} 
        %照片須放在和文檔相同之資料夾內
        \caption{霍斯特鏈懸掛} %有caption的圖才會編入目錄中
        \label{fig:scale} %此處的label相當於一個圖片的專屬標誌,目的是方便上下文的引用
    \end{figure} 
      \chapter{Pyslvs-UI 套件介紹}
      \section{Pyslvs-UI 架構與原理}
       \fourteen{Pyslvs-UI是一套利用Python3與PyQt5建立的平面機構模擬與合成系統。機構模擬與合成的主要核心包括Python-Solvespace幾何約束求解程式庫、三角幾何函式程式庫(tinycadlib)、演算程式庫(ADesign)、幾何約束求解程式庫(bgfs)、類型合成程式庫(topologic)、數目合成程式庫(number)等。其中,ADesign演算程式庫內包含Real-coded Genetic Algorithm (RGA)、Differential Evolution(DE)與Firefly Algorithm (Firefly)等三種,用於平面機構尺寸合成演算}
       
        \begin{figure}[H]
        \centering
        \includegraphics[scale=0.5]{Pyslvs-UI架構圖.png} 
        \caption{Pyslvs-UI架構圖} 
        \label{fig:scale}
    	\end{figure}
\newpage    	
      \subsection{平面連桿機構模擬}
      \begin{itemize} 
      \item Solvespace:以C ++語言編寫的開放原始碼電腦輔助設計軟件,將幾何約束求解器分為實體與約束兩部份。其中,實體:表示每個圖形零件的幾何數據。約束: 表示圖形零件之間的關係。

		\begin{figure}[H]
        \centering
        \includegraphics[scale=0.6]{Python-Solvespace求解流程.png} 
        \caption{Python-Solvespace流程圖} 
        \label{fig:scale}
    	\end{figure}
    	
      \item Sketch Solve:在BSD授權條款下,用C ++編寫的2D幾何求解器庫,原始碼僅使用表達式和函數來獲取答案。主要求解過程為:通過函數輸入創建約束對象,求解需要有參數列表、約束數據列表和步距精度,啟動BFGS算法後,參數將投影到每個實例,因此將更改值;若列表中沒有值,則該值將不會更改。
      	\end{itemize} 
      	
		\begin{figure}[H]
        \centering
        \includegraphics[scale=0.7]{Sketch slove.png} 
        \caption{Sketch slove流程圖} 
        \label{fig:scale}
    	\end{figure}
    	
      \fourteen{Cython是結合了Python和C語法的一種語言,使用者可以維持大部分的Python語法,而不需要大幅度調整主要的程式邏輯與演算法。}
      
      \subsection{機構合成}
      \fourteen{進行機構設計與合成主要流程分為數目合成、構造合成、尺寸合成與運動學模擬,合成皆有其順序性存在,而運動學模擬則是進行分析前面的合成項目。}
      \begin{itemize} 
      \item 數目合成:用來決定其設計機構的桿數與接頭數,如四連桿、五連桿、六連桿等,然而當機構的桿數與接頭數越多的時候,其組合變化亦越多,將所有可能的運動鏈全部列出來。
      \item 構造合成:構造合成可分為以下幾個步驟
      \end{itemize} 
      \begin{enumerate}
      \item 確定自由度數:機構的獨立輸入數是根據其工作目的來決定,為已知條件。
      \item 選擇運動空間類型:必須根據設計規範、設計要求、設計限制來選擇
      \item 決定接頭類型:接頭類型必須根據工作目的、運動空間類型來決定,若無特殊考量,以取自由度為1的接頭為宜。
      \item 求出機件與接頭數目:當機構的運動空間類型與接頭類型確定後,即可根據式(3.1)或(3.2)求出機件數目N與接頭數目J
      \hspace*{\fill} \\
      
      \begin{equation}
      F_{p}=3\left ( N-1 \right )-\sum J_{i}C_{pi}
      \end{equation}  
      
      \begin{equation}
      F_{s}=6\left ( N-1 \right )-\sum J_{i}C_{si}
      \end{equation}  
      \item 合成機構構造目錄:針對每一個具N根機件與J個接頭的運動鏈,根據要求與限制,將機件與接頭類型其分別配置到適合的桿件與接頭上。
      \hspace*{\fill} \\
      \end{enumerate}
      \begin{itemize} 
      \item 尺寸合成:
      \end{itemize} 
      \fourteen {通過隨機變量生成具有路徑限制的機構,生成結構參數來自變數設定,也有其它算法選擇,程式內共包含三種算法:實數編碼遺傳算法、螢火蟲算法與差分進化演算法。}
      \begin{enumerate}
      \item {實數編碼遺傳算法:在遺傳演算法的演算過程中,將設計變數以實數數值表示,稱為實數編碼遺傳演算法(RGA),其演算流程如圖3.1}
       \item {螢火蟲算法:其演算法是模擬螢火蟲離散的閃爍行為,算法將計算群體中螢火蟲的相對亮度和吸引度,並根據相對亮度決定螢火蟲的移動方向;更新螢火蟲的空間位置,對處在最佳位置的螢火蟲進行隨機移動;根據更新後螢火蟲位置,重新計算螢火蟲的亮度,最終輸出群體極值點和最優個體數值,演算流程如圖3.2。}
       \item {差分進化演算法:是一種求解最佳化問題的進化演算法,其算法]]]原理採用對個體進行方向擾動,以達到對個體的函式值進行下降的目的,同其他進化演算法一樣,差分進化演算法不利用函式的梯度資訊,因此對函式的可導性甚至連續性沒有要求,適用性很強,演算流程如圖3.3。} 
       \end{enumerate}
        \begin{figure}[hbt!]
        \centering
        \includegraphics[scale=0.4]{實數編碼遺傳算法.png} 
        \caption{實數編碼遺傳算法流程圖}
        \label{fig:scale}
        \end{figure}
        
        \begin{figure}[hbt!]
        \centering
        \includegraphics[scale=0.4]{螢火蟲算法.jpg} 
        \caption{螢火蟲算法流程圖}
        \label{fig:scale}
        \end{figure}
        
        \begin{figure}[H]
        \centering
        \includegraphics[scale=0.4]{差分進化演算法.jpg} 
        \caption{差分進化演算法流程圖}
        \label{fig:scale}
    \end{figure}
    \newpage
      \subsection{圖形化使用者介面}
      \begin{itemize}
      \item{PyQt}
      \end{itemize}
      \fourteen {Qt被廣泛應用於開發圖形化介面程式。主要特色為使用相同的程式碼皆可在支援的平台上執行與編譯,PyQt是Python語言的GUI編程解決方案之一,與Qt一樣是一個自由軟體,其API也與Qt類似,因此Qt的文件仍然可以應用於此,且與相同性質的軟體比較,PyQt的編程庫文件豐富許多。} 
      \begin{figure}[hbt!]
        \centering
        \includegraphics[scale=0.1]{PyQt.png} 
        \caption{PyQt圖示}
        \label{fig:scale}
        \end{figure}
      \section{Pyslvs-UI 編譯}
       \lstset{language=Python, %設定語言
		basicstyle=\footnotesize, 
		frame=lines,
		caption={Python code}, %設定capion
		label={lst:example} %設定label
		}



\subsection{環境與安裝套件}
		\begin{itemize}
		\item GIT:利用以下兩種系統進行遠端檔案版本控制的工具。
		\hspace*{\fill} \\
		Ubuntu:\\
		\begin{lstlisting}[caption=使用apt安裝Git]
$sudo apt install git
		\end{lstlisting}
		\hspace*{\fill} \\
		Windows:\\
		\begin{lstlisting}[caption=下載新版Git]
https://git-scm.com/download/win
		\end{lstlisting}
		\end{itemize}
     
     \begin{itemize}
		\item Cython:為了增進運算速度將Python語言轉換成C++的套件,並使用 pip 安裝 Cython。\\
		\begin{lstlisting}[caption=安裝 Cython]
$sudo pip3 install cython 
>pip install cython
		\end{lstlisting}
		\end{itemize}
		
	\begin{itemize}
		\item Windows編譯工具:使用Windows中的GNU工具包以操作Make功能。
		\begin{lstlisting}[caption=下載新版MinGW(32Bit)]
https://sourceforge.net/projects/mingw-w64/files/latest/download?source=files
		\end{lstlisting}
		另外必須安裝Microsoft Visual Studio以編譯64位元的Cython程式碼,且安裝Windows tool kit。
		\begin{lstlisting}[caption=安裝各類編譯程式]
https://www.visualstudio.com/downloads/
		\end{lstlisting}
		\end{itemize}
		   
	\begin{itemize}
		\item SWIG:將C或C++配上自訂接口後,可變成其他高階語言函式庫的套件。
		\hspace*{\fill} \\
		Ubuntu:\\
		\begin{lstlisting}[caption=使用apt安裝]
$sudo apt install swig
		\end{lstlisting}
		Windows:\\
		\begin{lstlisting}[caption=下載新版SWIG]
http://www.swig.org/download.html
		\end{lstlisting}
		解壓縮到安裝位置,並將該資料夾加入環境變數。
		\begin{lstlisting}[caption=安裝位置]
C:\swigwin-3.0.12
		\end{lstlisting}
		\end{itemize}
		 
	\begin{itemize}
		\item PyQt:於官方網站上下載Qt安裝檔,並選擇PyQt有所對應的新版本,以免API對應不上。
		\begin{lstlisting}[caption=下載Qt檔]
https://www.qt.io/download/
		\end{lstlisting}
		Ubuntu:\\
		\begin{lstlisting}[caption=安裝後將Qt函式庫位址加入~/.bashrc環境變數]
export QTDIR=/opt/Qt/5.8/gcc_64/
export LD_LIBRARY_PATH=${LD_LIBRARY_PATH}:${QTDIR}/lib/
export PATH=${QTDIR}/bin:${PATH}
		\end{lstlisting}
		Windows:安裝後將Qt函式庫位址加入環境變數,再另外安裝PyQt\\
		\begin{lstlisting}[caption=Qt函式庫位址加入環境變數]
C:\Qt\5.8\msvc2015_64\bin。
		\end{lstlisting}
		\end{itemize}       
      
     \begin{itemize}
		\item Ubuntu:由於某部分無法使用pip,需先確認並刪除Python套件安裝位置的SIP檔案
		\begin{lstlisting}[caption=刪除SIP檔案]
$sudo rm -f /usr/lib/python3/dist-packages/sip*
		\end{lstlisting}
		\begin{lstlisting}[caption=下載新版SIP]
https://riverbankcomputing.com/software/sip/download
		\end{lstlisting}
		\begin{lstlisting}[caption=解壓縮後安裝]
$python3 configure.py
$sudo make install
		\end{lstlisting}
		\begin{lstlisting}[caption=核對Python內的版本]
$sip -V
>>import sip
>>print(sip.SIP_VERSION_STR)
		\end{lstlisting}
		\begin{lstlisting}[caption=下載新版PyQt5]
https://riverbankcomputing.com/software/pyqt/download5
		\end{lstlisting}
		\begin{lstlisting}[caption=解壓縮後安裝]
$python3 configure.py
$sudo make install
		\end{lstlisting}
		\begin{lstlisting}[caption=下載新版的QScintilla 2]
https://riverbankcomputing.com/software/qscintilla/download
		\end{lstlisting}
		
		\begin{lstlisting}[caption=本體]
$cd Qt4Qt5
$qmake qscintilla.pro
$sudo make install
		\end{lstlisting}
		\begin{lstlisting}[caption=Designer]
$cd designer-Qt4Qt5
$qmake designer.pro
$sudo make install
		\end{lstlisting}
		\begin{lstlisting}[caption=Python bundings]
$cd Python
$python3 configure.py --pyqt=PyQt5
$sudo make install
		\end{lstlisting}
		\end{itemize}  
		     
     \begin{itemize}
		\item Windows:
		\begin{lstlisting}[caption=下載Qt檔]
https://www.qt.io/download/
		\end{lstlisting}
		Ubuntu:\\
		\begin{lstlisting}[caption=使用pip安裝PyQt5]
>pip install pyqt5 -U
		\end{lstlisting}
		\begin{lstlisting}[caption=使用pip安裝QScintilla 2]
>pip install qscintilla
		\end{lstlisting}
		\end{itemize} 
		      
     \begin{itemize}
		\item Python編譯工具:將Python程式碼編譯成可執行檔的工具。
		\hspace*{\fill} \\
		Ubuntu:使用PyInstaller即可。
		\begin{lstlisting}[caption=安裝PyInstaller]
$sudo pip3 install PyInstaller
		\end{lstlisting}
		Windows:使用PyInstaller編譯主體,再用cxFreeze包裝函式庫。\\
		\begin{lstlisting}[caption=安裝cxFreeze]
>pip install cx_Freeze PyInstaller
		\end{lstlisting}
		如果使用官方來源有編碼錯誤,可以用下面的倉儲:
		\begin{lstlisting}[caption=下載倉儲]
>pip install
 https://github.com/pyinstaller/pyinstaller/archive/develop.zip
		\end{lstlisting}
		\end{itemize} 
		
	\begin{itemize}
		\item Eric IDE:若要進行程式協同,需安裝此Python編輯器。
		\begin{lstlisting}[caption=下載Eric IDE]
http://eric-ide.python-projects.org/eric-download.html
		\end{lstlisting}
		\begin{lstlisting}[caption=執行install.py]
$sudo python3 install.py
>python install.py
		\end{lstlisting}
		\end{itemize}
       

      
    
      \section{Pyslvs-UI 範例}
		\subsection{Crank Lifter (曲柄滑塊機構)}
       \hspace*{\fill} \\
       \fourteen {主要以6個連桿、7個接頭,4個雙接頭連桿、1個三接頭連桿所構成,如圖(3.1),由四連桿機構衍伸而來多樣性運動型態、軌跡,可滿足一般運動需求,常用於汽車工業中加速與製動,從踏板的輸入到化油器、製動缸、自行車剎車都廣泛運用在其中,如圖(3.2)。}
     \begin{figure}[H]
        \centering
        \includegraphics[scale=0.5]{曲柄滑塊機構.png} 
        %照片須放在和文檔相同之資料夾內
        \caption{曲柄滑塊機構} %有caption的圖才會編入目錄中
        \label{fig:scale} %此處的label相當於一個圖片的專屬標誌,目的是方便上下文的引用
    \end{figure}
    \begin{figure}[H]
        \centering
        \includegraphics[scale=0.6]{曲柄滑塊機構運用範圍.png} 
        %照片須放在和文檔相同之資料夾內
        \caption{曲柄滑塊機構運用範圍} %有caption的圖才會編入目錄中
        \label{fig:scale} %此處的label相當於一個圖片的專屬標誌,目的是方便上下文的引用
    \end{figure}
    
    
       \hspace*{\fill} \\
       \subsection{Crank Rocker(曲柄搖桿機構)}
       \hspace*{\fill} \\        
        \fourteen {為4個連桿、5個接頭,2個雙接頭連桿、1個三接頭連桿所組成,如圖(5.3),曲柄旋轉運動透過連接桿轉換至搖桿的搖擺運動,此四連桿機構構造最簡單也最常在生活周遭出現,如:老虎鉗、雷達調整機構、縫紉機腳踏機構、鋼材輸送機,如圖(5.4)。}
     \begin{figure}[H]
        \centering
        \includegraphics[scale=0.5]{曲柄搖桿機構.png} 
        %照片須放在和文檔相同之資料夾內
        \caption{曲柄搖桿機構} %有caption的圖才會編入目錄中
        \label{fig:scale} %此處的label相當於一個圖片的專屬標誌,目的是方便上下文的引用
    \end{figure}
    \begin{figure}[H]
        \centering
        \includegraphics[scale=0.6]{曲柄搖桿機構範例.png} 
        %照片須放在和文檔相同之資料夾內
        \caption{曲柄搖桿機構範例} %有caption的圖才會編入目錄中
        \label{fig:scale} %此處的label相當於一個圖片的專屬標誌,目的是方便上下文的引用
    \end{figure}
    
    
    	\hspace*{\fill} \\
       \subsection{Stephenson(史蒂芬生機構)}
       \hspace*{\fill} \\        
        \fourteen {為6個連桿、7個接頭,3個雙接頭連桿、2個三接頭連桿所組成,使三接頭連桿作為固定桿,其中一雙接頭連桿作曲柄運動,其餘桿件與三接頭連桿一起做搖擺運動,如圖(3.5),此機構僅次於四連桿組的重要單自由度連桿機構也稱作複合連桿組機構,常用於自動化夾緊機構、越野摩托車後懸吊系統,如圖(3.6)。
}
     \begin{figure}[H]
        \centering
        \includegraphics[scale=0.5]{史蒂芬生機構作動.png} 
        %照片須放在和文檔相同之資料夾內
        \caption{史蒂芬生機構作動} %有caption的圖才會編入目錄中
        \label{fig:scale} %此處的label相當於一個圖片的專屬標誌,目的是方便上下文的引用
    \end{figure}
    \begin{figure}[H]
        \centering
        \includegraphics[scale=0.4]{史蒂芬生機構範例.jpg} 
        %照片須放在和文檔相同之資料夾內
        \caption{史蒂芬生機構範例} %有caption的圖才會編入目錄中
        \label{fig:scale} %此處的label相當於一個圖片的專屬標誌,目的是方便上下文的引用
    \end{figure}
    
    
    \hspace*{\fill} \\
       \subsection{Watt(瓦特氏運動機構)}
       \hspace*{\fill} \\        
        \fourteen {由6個連桿、7個接頭,4個雙接頭連桿、1個三接頭連桿構成,本身機械鏈構型為上下左右對稱,使三接頭桿為固定桿另其中一支雙接頭桿為曲柄作旋轉運動,其他桿件則是搖擺運動,如圖(3.7),常用於自動化平台舉高移送機構、自動化包裹移送機構、磨耗測試裝置、飛機起落收放機構,如圖(3.8)。}
     \begin{figure}[H]
        \centering
        \includegraphics[scale=0.5]{瓦特氏運動機構.png} 
        %照片須放在和文檔相同之資料夾內
        \caption{瓦特氏運動機構} %有caption的圖才會編入目錄中
        \label{fig:scale} %此處的label相當於一個圖片的專屬標誌,目的是方便上下文的引用
    \end{figure}
    \begin{figure}[H]
        \centering
        \includegraphics[scale=0.6]{瓦特氏運動機構飛機起落收放.png} 
        %照片須放在和文檔相同之資料夾內
        \caption{瓦特氏運動機構飛機起落收放} %有caption的圖才會編入目錄中
        \label{fig:scale} %此處的label相當於一個圖片的專屬標誌,目的是方便上下文的引用
    \end{figure}
    
    
  	 \hspace*{\fill} \\
       \subsection{Double Ball Lifter(二段式提球機構)}
       \hspace*{\fill} \\        
        \fourteen {由兩個史蒂芬生 (Stephenson) 機構衍伸出來,如圖(3.9),將機構融合調整位置後為10個連桿、14個接頭,8個雙接頭連桿、1個三接頭連桿所組成,如圖(3.10),便於機構執行時接近置球點底部,將球提起並筆直向上提升與下一個提球機構交會完成傳接行程,常運用在搖臂式發球機上,如圖(3.11)。}
    \begin{figure}[H]
        \centering
        \includegraphics[scale=0.6]{雙球升降器.png} 
        %照片須放在和文檔相同之資料夾內
        \caption{雙球升降器} %有caption的圖才會編入目錄中
        \label{fig:scale} %此處的label相當於一個圖片的專屬標誌,目的是方便上下文的引用
    \end{figure}
    \begin{figure}[H]
        \centering
        \includegraphics[scale=0.3]{搖臂式發球機.jpg} 
        %照片須放在和文檔相同之資料夾內
        \caption{搖臂式發球機} %有caption的圖才會編入目錄中
        \label{fig:scale} %此處的label相當於一個圖片的專屬標誌,目的是方便上下文的引用
    \end{figure}
        \chapter{登山車避震機構}
      \section{避震機構合成}
      \quad \fourteen {機構合成設計需求為騎乘姿勢改變時或遇到顛頗不整的路時仍能保持整體的平衡,即Anti-squat及Anti-rise的數值能在100\%附近,而上述合成需求之適應函數可表示為}
      \hspace*{\fill} \\
      
      \begin{equation}
      h_{2} = r_{of}\left (\frac{r_{ppy}-IFC_{y}}{r_{ppx}-IFC_{x}}\right ) + \left[ r_{ppy}-r_{ppx}\left( \frac{r_{ppy}-IFC_{y}}{r_{ppx}-IFC_{x}} \right) \right]
      \end{equation}  
      \hspace*{\fill} \\
      
      \fourteen {其中}
      \begin{equation}
      IFC_{x} = \frac{\left[ P_{2y} - \left( \frac{P_{2y} - IC_{y}}{P_{2x} - IC_{x}} \right )P_{2x} - otl_{b} \right ]}{\frac{P_{2y} - IC_{y}}{P_{2x} - IC_{x}}-otl_{a}}
      \end{equation}
      \hspace*{\fill} \\
      

      \begin{flushleft}
      \begin{equation}
      IFC_{y} = otl_{a} \cdot \left( IFC_{x} \right ) + otl_{b}
      \end{equation}
      \end{flushleft}
      \hspace*{\fill} \\
      
      \begin{equation}
      IC_{x} = \frac{P_{3y} - \left [ \frac {(P_{3y} - P_{4y})P_{3x}}{P_{3x}-P_{4x}} \right ]}{\frac{P_{1y}}{P_{1x}}-\left ( \frac{P_{3y} - P_{4y}}{P_{3x}-P_{4x}} \right )}
      \end{equation}
      \hspace*{\fill} \\
      
      \begin{equation}
      IC_{y} = \left ( \frac{P_{1y}}{P_{1x}} \right ) IC_{x}
      \end{equation}
      \hspace*{\fill} \\
      
      \fourteen {
      	\begin{flushleft}
      	$r_{ppx}$: 後輪與地面接觸點到座標原點的x座標
      	\\
      	\hspace*{\fill} \\
      	$r_{ppy}$: 後輪與地面接觸點到座標原點的y座標
      	\\
      	\hspace*{\fill} \\
      	$P_{nx}$: 點n的x座標
      	\\
      	\hspace*{\fill} \\
      	$P_{ny}$: 點n的y座標
      	\\
      	\hspace*{\fill} \\
      	$r_{of}$: 座標原點到前輪樞紐的水平距離
      	\end{flushleft}
      	}
      \hspace*{\fill} \\
      \section{避震機構評量}
      \subsection{Anti-squat}
      \fourteen {當加速時因不同騎乘姿勢導致的重心位置變化而產生的力量轉移會讓懸架受壓縮力,而Anti-squat即為抵抗此後沉的能力。}
       \begin{itemize}
       \item > 100\%: 表示足以抵抗後沉的力量,而超過的力量將會變成拉伸力
       \item 100\%: 表示完全平衡了負載,即懸吊沒受到壓縮力或拉伸力
       \item 0~100\%: 表示部分後沉力量被抵抗
       \item < 0\%: 表示所有的後沉力量皆無被抵抗
       \end{itemize}
       \hspace*{\fill} \\
       \subsection{Anti-rise}
       \fourteen {在剎車的時候因為力量轉移的關係導致後輪產生升起的情況,而Anti-rise即為抵抗此後升現象的能力。}
       \begin{itemize}
       \item > 100\%: 表示足以抵抗後升的力量,而超過的力量將會變成壓縮力
       \item 100\%: 表示完全抵抗了剎車產生的後升現象
       \item 0~100\%: 表示部分後升力量被抵抗
       \item < 0\%: 表示所有的後升力量皆無被抵抗
       \end{itemize}
       \hspace*{\fill} \\
       \subsection{Leverage Ratio}
       \fourteen {Leverage ratio是避震器壓縮量和後輪行程的比值,當較大的槓桿比率會對避震器產生較大的衝擊,而在騎乘時對於地形變化的感受度較不敏感,相反地較小的槓桿比率對於地形的敏感度較高。其計算方式如下所示。}
       \hspace*{\fill} \\
       \begin{equation}
       Leverage\, Ratio = \frac{wheel\,travel}{stroke}
       \end{equation}
       \hspace*{\fill} \\
      \section{避震機構分析範例}
		\fourteen {此小節會將章節一所提到的自行車避震機構類型為範例,說明在不同避震機構下經由尺寸合成後得到的評比結果。此評比結果會以自行車後輪的垂直上升行程作為變數來探討Anti-squat以及Anti-rise的表現。
      \hspace*{\fill} \\
      \hspace*{\fill} \\
      \newpage
      \fourteen {圖\ref{fig_Horst-link:scale}為Horst-link類型並以Niner RDO 29作為該類型自行車範例在後輪行程0到150mm的Anti-squat及Anti-rise的表現。結果得出Horst-link類型在Anti-squat的表現會隨著後輪行程的上升而下降,較無維持在100\%的現象,而Anti-rise雖下降趨勢不明顯但數值僅維持在50\%較並不算特別突出的表現。}
      \begin{figure}[hbt!]
        \centering
        \includegraphics[scale=0.6]{Horst_link.png}
        \caption{Horst-link suspension}
        \label{fig_Horst-link:scale}
    	\end{figure}
    	
    \hspace*{\fill} \\
    \fourteen {圖\ref{fig_Twin-link:scale}為Twin-link類型並以Pivot Firebird 29作為該類型自行車範例在後輪行程0到150mm的Anti-squat及Anti-rise的表現。結果得出Twin-link類型在Anti-squat的表現在後輪垂直上升行程到100mm的時候還能保持在90\%的效益接著才會有明顯下降的趨勢,而Anti-rise雖只能保持在約70\%的效益,但是降低的趨勢相較於Horst-link類型更為趨緩。}
    
      \begin{figure}[hbt!]
        \centering
        \includegraphics[scale=0.6]{Twin-link.png}
        \caption{Twin-link suspension}
        \label{fig_Twin-link:scale}
    	\end{figure}    	
    	
        \chapter{結論}
        {\begin{flushleft}
        \quad \quad \fourteen {對於科技的迅速進步,開源已成了當今趨勢,因此本專題所用的套件不僅能達到其他平面機構套件大部分所擁有的功能但又有公開原始碼的優勢因此在後續的發展應該會比相對封閉的套件更加快速。而在CAD當中常有許多相似專案但是每次繪製不同零件時又要從頭開始繪製的情形,此時若沒有一套程式流程將會產生高耗時的問題,因此在本專題中所建立的自動3D零件生成以及後續利用xml檔完成組立、碰撞設定、馬達轉軸輸出設定......再直接匯入至動態分析套件減少了很多我們在進行不同機構範例分析的時間。後續Pyslvs-UI若能有更多的機構類型或是做後續的CAE分析應該能有更完整的設計流程。}
        \end{flushleft}

        \chapter{未來研究建議}
        {\begin{flushleft}
        \fourteen {本專題以已建立一流程從尺寸合成至動態模擬軟體進行運動模擬,後續可將運動系統改為動力系統並利用有限元素法對該避震機構進行受力情況的分析。}
        \begin{figure}[hbt!]
        \centering
        \includegraphics[scale=0.6]{follow-up research.png}
        \caption{後續研究}
        \label{fig:scale}
    	\end{figure}
        \end{flushleft}
          
        \newpage
        \addcontentsline{toc}{chapter}{\fourteen{參考文獻}}
	\begin{center}
	\twenty \bf{參考文獻}\\
	\end{center}
	\begin{flushleft}
	\begin{enumerate}[1.]
	\item \quad \fourteen {"How your bike’s suspension design affects its performance on the trail" \quad https://enduro-mtb.com/en/mtb-suspension-systems/}
	\item \quad \fourteen {"What Is Anti-Squat \& How Does It Actually Affect Mountain Bike Performance" \quad https://www.pinkbike.com/news/definitions-what-is-anti-squat.html}
	\item \quad \fourteen {"Suspension Design Theory and Anti-Squat Behavior" \quad http://blog.artscyclery.com/mountain/suspension-design-theory-and-anti-squat-behavior/}
	\item \quad \fourteen {"How to Calculate Leverage Ratios" \quad https://mbaction.com/garage-files-leverage-ratio/}
	\item \quad \fourteen {"Suspension Tech: What is Anti-Rise?" \quad https://bikerumor.com/2018/10/04/suspension-tech-what-is-anti-rise/}
	\end{enumerate}
	
	
	% [2] \quad \fourteen {"What Is Anti-Squat & How Does It Actually Affect Mountain Bike Performance" \quad https://www.pinkbike.com/news/definitions-what-is-anti-squat.html}
	
    \end{flushleft}
    \newpage
    
    \addcontentsline{toc}{chapter}{\fourteen{誌謝}}
    \begin{center}
	\twenty \bf{誌謝}\\
	\end{center}

	\fourteen {德國詩人歌德曾說過:「我不應把我的作品全歸功於自己的智慧,還應歸功於我以外向我提供素材的成千成萬的事情和人物。」
	\hspace*{\fill} \\
	本研究之所以可以順利完成,首先感謝負責指導的專題教授嚴家銘老師,老師細心的指導與適時的提供建議下,確立整體研究架構與目標,更讓我們深入了解平面多連桿機構套件的理論,並在研究過程中培養我們解決問題以及獨立思考的能力。
	\hspace*{\fill} \\
	再來要感謝共同指導教授李武鉦老師,運用課堂所教導的Python程式語法,使我們在專題研究中,得以延伸學習與應用,跳脫僵化的思維去探討所要面對之問題。
	\hspace*{\fill} \\
	最後感謝學校提供完善的設備與資源,使本專題能夠圓滿完成;在專題研究上,系上提供軟硬體設備以及專題實驗室,讓我們可以利用空堂及課後時間製作與討論專題。}

    \newpage
 
    \addcontentsline{toc}{chapter}
   	{\fourteen{作者簡介}}
    \begin{center}
	\twenty \bf{作者簡介}\\
	\end{center}
	
	{\begin{textblock}{6}(0,0.5)
	\begin{figure}
	\includegraphics[width=1.15in]{st1.jpg} 
	\end{figure}
	\end{textblock}}
	{\renewcommand\baselinestretch{0.99}\selectfont %設定以下行距
	{\begin{textblock}{15}(3.5,0.7)%{寬度}(以左上角為原點之右移量,下移量)
	\noindent\fourteen \makebox[4em][s]{姓名}\enspace:\enspace
    \fourteen \makebox[4em][s]{林昱秀}\\     \hspace*{\fill} \\
    \fourteen \makebox[4em][s]{學號}\enspace:\enspace
    \fourteen \makebox[4em][s]{40723102} \\ %\makebox為文本盒子
    \hspace*{\fill} \\
    \fourteen \makebox[4em][s]{畢業學校}\enspace:\enspace
    \fourteen \makebox[9em][s]{國立虎尾科技大學}\\
    \fourteen \makebox[5em][s]{\quad}\enspace\enspace
    \fourteen \makebox[8em][s]{機械設計工程系}\\
    \hspace*{\fill} \\
    \fourteen \makebox[4em][s]{經歷}\enspace:\enspace
    \end{textblock}}}
    \hspace*{\fill} \\

	{\begin{textblock}{6}(0,2.3)
	\begin{figure}
	\includegraphics[width=1.15in]{st2.jpg} 
    \end{figure}
    \end{textblock}}
    {\renewcommand\baselinestretch{0.99}
    \selectfont %設定以下行距
    {\begin{textblock}{15}(3.5,2.5) %{寬度}(以左上角為原點之右移量,下移量)
\noindent\fourteen \makebox[4em][s]{姓名}\enspace:\enspace
\fourteen \makebox[4em][s]{林晏瑩}\\ 
\hspace*{\fill} \\
\fourteen \makebox[4em][s]{學號}\enspace:\enspace
\fourteen \makebox[4em][s]{40723103} \\ 
\hspace*{\fill} \\
\fourteen \makebox[4em][s]{畢業學校}\enspace:\enspace
\fourteen \makebox[9em][s]{國立虎尾科技大學}\\
\fourteen \makebox[5em][s]{\quad}\enspace\enspace
\fourteen \makebox[8em][s]{機械設計工程系}\\
\hspace*{\fill} \\
\fourteen \makebox[4em][s]{經歷}\enspace:\enspace
    \end{textblock}}}
    \hspace*{\fill} \\


    {\begin{textblock}{6}(0,4.1)
    \begin{figure}
        \includegraphics[width=1.15in]{st3.jpg} 
    \end{figure}
    \end{textblock}}
    {\renewcommand\baselinestretch{0.99}\selectfont %設定以下行距
    {\begin{textblock}{15}(3.5,4.3) %{寬度}(以左上角為原點之右移量,下移量)
\noindent\fourteen \makebox[4em][s]{姓名}\enspace:\enspace%\noindent指定首行不進行縮排
\fourteen \makebox[4em][s]{劉光智}\\ 
\hspace*{\fill} \\
\fourteen \makebox[4em][s]{學號}\enspace:\enspace
\fourteen \makebox[4em][s]{40723145} \\ %\makebox為文本盒子
\hspace*{\fill} \\
\fourteen \makebox[4em][s]{畢業學校}\enspace:\enspace
\fourteen \makebox[9em][s]{國立虎尾科技大學}\\
\fourteen \makebox[5em][s]{\quad}\enspace\enspace
\fourteen \makebox[8em][s]{機械設計工程系}\\
\hspace*{\fill} \\
\fourteen \makebox[4em][s]{經歷}\enspace:\enspace
    \end{textblock}}}
    \hspace*{\fill} \\

    {\begin{textblock}{6}(0,5.9)
    \begin{figure}
        \includegraphics[width=1.15in]{st4.jpg} %{}內是圖片文件的相對路徑
    \end{figure}
    \end{textblock}}
    {\renewcommand\baselinestretch{0.99}\selectfont %設定以下行距
    {\begin{textblock}{15}(3.5,6.1) %{寬度}(以左上角為原點之右移量,下移量)
\noindent\fourteen \makebox[4em][s]{姓名}\enspace:\enspace
\fourteen \makebox[4em][s]{吳佳穎}\\ \hspace*{\fill} \\
\fourteen \makebox[4em][s]{學號}\enspace:\enspace
\fourteen \makebox[4em][s]{40723153} \\ \hspace*{\fill} \\
\fourteen \makebox[4em][s]{畢業學校}\enspace:\enspace
\fourteen \makebox[9em][s]{國立虎尾科技大學}\\
\fourteen \makebox[5em][s]{\quad}\enspace\enspace
\fourteen \makebox[8em][s]{機械設計工程系}\\
\hspace*{\fill} \\
\fourteen \makebox[4em][s]{經歷}\enspace:\enspace
    \end{textblock}}}
    \hspace*{\fill} \\

    {\begin{textblock}{6}(0,7.7)
    \begin{figure}
        \includegraphics[width=1.15in]{st5.jpg} %{}內是圖片文件的相對路徑
    \end{figure}
    \end{textblock}}
    \renewcommand\baselinestretch{0.99}\selectfont %設定以下行距
    {\begin{textblock}{15}(3.5,7.9) %{寬度}(以左上角為原點之右移量,下移量)
	\noindent\fourteen \makebox[4em][s]{姓名}\enspace:\enspace
	\fourteen \makebox[4em][s]{蔡育灃}\\ \hspace*{\fill} \\
	\fourteen \makebox[4em][s]{學號}\enspace:\enspace
	\fourteen \makebox[4em][s]{40723245} \\ \hspace*{\fill} \\
	\fourteen \makebox[4em][s]{畢業學校}\enspace:\enspace
	\fourteen \makebox[9em][s]{國立虎尾科技大學}\\
	\fourteen \makebox[5em][s]{\quad}\enspace\enspace
	\fourteen \makebox[8em][s]{機械設計工程系}\\
	\hspace*{\fill} \\
	\fourteen \makebox[4em][s]{經歷}\enspace:\enspace
    \end{textblock}}
\end{document}